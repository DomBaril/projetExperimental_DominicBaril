\documentclass[12pt]{article}

\usepackage[utf8]{inputenc}
\usepackage[printonlyused]{acronym}
\usepackage{biblatex}
\usepackage[margin2.0cm]{geometry}

\title{\vspace{-2.0cm} Projet expérimental - IFT-7026}
\author{\textbf{Dominic Baril} \\ \small Supervisé par François Pomerleau}
\date{\vspace{-5ex}}

\addbibresource{references.bib}

\acrodef{CTU Prague}{Université technique de Prague}
\acrodef{ICP}{Iterative Closest Point}
\acrodef{UGV}{Véhicule terrestre sans pilote}

\begin{document}
    \maketitle
    
     En Février 2020, une équipe du laboratoire Norlab de l'Université Laval collaborera avec une équipe de l'\ac{CTU Prague} pour participer au \textit{DARPA Subterranean Challenge}\footnote{\url{https://www.darpa.mil/program/darpa-subterranean-challenge}}, une compétition internationnale de robotique mobile. 
     Les membres de l'équipe du Norlab seront responsables de l'implémentation d'un algorithme de perception tri-dimensionnelle fonctionnant avec des nuages de points produits par des capteurs lidars.
     
     \bigskip
     
     La perception est un champ majeur de la robotique mobile car il permet aux robots de se localiser dans l'environnement.
     Pour ce faire, il est possible d'aligner des nuages de points pris à plusieurs endroits sur une trajectoire. 
     L'algorithme le plus populaire dans la littérature pour effectuer cette tâche est \ac{ICP} \cite{Chen1991}. 
     Cet algorithme peut être intégré à un robot mobile afin de lui permettre de naviguer dans un environnement inconnu et de le cartographier de manière précise en temps réel \cite{Pomerleau2013, Pomerleau2015}. %(source path planning?)
     
     \bigskip
     
     La projet vise donc à intégrer un \ac{UGV} pour l'ajouter à la flotte hétérogène de véhicules autonomes de \ac{CTU Prague}. 
     Cette plateforme sera un \textit{Husky}\footnote{\url{https://clearpathrobotics.com/husky-unmanned-ground-vehicle-robot/}} de la compagnie Clearpath Robotics, auquel un capteur lidar \textit{VLP-16}\footnote{\url{https://velodynelidar.com/products/puck/}} de Velodyne sera ajouté pour produire des nuages de points. 
     La librairie \textit{libpointmatcher}\footnote{\url{https://github.com/ethz-asl/libpointmatcher}} sera utilisée afin de produire la localisation du robot et les cartes de l'environnement en temps réel. 
     Ces cartes serviront à identifier la localisation d'objectifs situés dans le parcours de la compétition. 
     L'exactitude de cette identification détermine le pointage de la compétition.
     
     \bigskip
     
     Suite à la compétition, les résultats obtenus seront comparés à la vérité terrain fournie par les organisateurs afin d'identifier les erreurs. 
     Des jeux de données récoltés lors de la compétition seront utilisés afin de faire fonctionner l'algorithme \ac{ICP} dans les mêmes conditions une fois celle-ci terminée.
     Une analyse des sources d'erreur sera réalisée et une proposition des paramètres à modifier sera réalisée. 
     Un rapport technique sera produit pour le partenaire du projet, General Dynamics, en tant que livrable final.
     
     \newpage
     \printbibliography[title=Références]
    
\end{document}
