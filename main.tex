\documentclass[12pt]{article}

\usepackage[utf8]{inputenc}
\usepackage[printonlyused]{acronym}
\usepackage{biblatex}
\usepackage[margin2.0cm]{geometry}

\title{\vspace{-2.0cm} Projet expérimental - IFT-7026}
\author{\textbf{Dominic Baril} \\ \small Supervisé par François Pomerleau}
\date{\vspace{-5ex}}

\addbibresource{references.bib}

\acrodef{CTU Prague}{Université technique de Prague}
\acrodef{ICP}{\textit{Iterative Closest Point}}
\acrodef{UGV}{\textit{Unmanned Ground Vehicle}}

\begin{document}
    \maketitle
     La perception est un champ majeur de la robotique mobile car il permet aux robots de se localiser dans l'environnement.
     Pour ce faire, il est possible d'aligner des nuages de points pris à plusieurs endroits sur une trajectoire. 
     L'algorithme le plus populaire dans la littérature pour effectuer cette tâche est \ac{ICP} \cite{Chen1991}. 
     Cet algorithme peut être intégré à un robot mobile afin de lui permettre de naviguer dans un environnement inconnu et de le cartographier de manière précise en temps réel \cite{Pomerleau2013, Pomerleau2015}. 
     
     \bigskip
     
     Le projet vise à évaluer la capacité d'un \ac{UGV} de cartographier un environnement intérieur.
     Cette plateforme sera un \textit{Husky}\footnote{\url{https://clearpathrobotics.com/husky-unmanned-ground-vehicle-robot/}} de la compagnie Clearpath Robotics, auquel un capteur lidar \textit{RS-LiDAR-16}\footnote{\url{https://www.robosense.ai/rslidar/rs-lidar-16}} de Robosense sera ajouté pour produire des nuages de points. 
     La librairie \textit{libpointmatcher}\footnote{\url{https://github.com/ethz-asl/libpointmatcher}} sera utilisée afin de produire la localisation du robot et les cartes de l'environnement grâce à l'algorithme \ac{ICP}. 
     
     
     \bigskip
     
     Afin d'évaluer la capacité de la plateforme pour effectuer cette tâche, deux environnements d'essais seront utilisés. 
     Le premier est le rez-de-chaussé du pavillon Adrien Pouliot de l'Université Laval. 
     Le deuxième est une usine désaffectée qui sert de parcours pour la compétition \textit{DARPA Subterranean Challenge}\footnote{\url{https://www.darpa.mil/program/darpa-subterranean-challenge}}, qui se tient aux États-Unis en février 2020.
     Un ensemble optimal de paramètres de l'algorithme devront être sélectionnés afin de lui permettre de produire une carte précise sur une trajectoire de plusieurs centaines de mètres.
     
     \bigskip
     
     L'algorithme ICP sera utilisé hors ligne sur les jeux de données préalablement récoltés dans ces deux environnements. 
     L'exactitude des cartes formées grâce aux données récoltées à l'Université sera évaluée via la distance de fermeture de boucle.
     L'exactitude des cartes formées via les données récoltées à la compétition sera plutôt évaluée en comparant les cartes à la vérité terrain qui sera fournie par les organisateurs.
     Un rapport technique contenant une analyse des sources d'erreur et une proposition des paramètres à modifier sera réalisé.
     
     \newpage
     \printbibliography[title=Références]
    
\end{document}
